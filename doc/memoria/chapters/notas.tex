

There is some analogy between the exploration and exploitation of the materials of nature in chalcolithic times and earlier, the detailed exploration of the forms of nature that followed increased representational skill in the 13th and 14th centuries, and the experiments with perspective, light, and shadow in the Italian Renaissance. The driving force in all three was an essentially scientific curiosity directed to the discovery of some fairly practical means of achieving an aesthetic end.

% Smith, Cyril Stanley. “Art, Technology, and Science: Notes on Their Historical Interaction.” Vol. 11, no. 4, 1970, pp. 493–549., doi:10.2307/3102690. Accessed 17 Apr. 2018. 

* * *

The impact of technology on contemporary life and culture is a vital issue in our age. Critical theory and cultural studies attempt to link the arts, literature, media studies, politics, sociology, anthropology, philosophy, and technology in an interdisciplinary search for relevant concepts and frameworks with which to understand the current world. While art practice and theory are being radically reshaped by this activity, the techno-scientific world in general has not deeply engaged the concepts from cultural studies. 

[...]

Artistscan participate in the cycle of research, invention, and development in many ways. They can learn enough to become researchers and inventors themselves. The claim that a unification is now impossible because scientific or technological research requires mastery of too much specialized knowledge and access to an elaborate research requires mastery of too much specialized knowledge and access to an elaborate research infraestructure must be critically scrutinized.

Free from the demands of the market and the socialization of particular disciplines, artists can explore and extend principles and technologies in unanticipated ways. They can pursue "unprofitable" lines of inquiry or research outside of disciplinary priorities. They can integrate disciplines and create events that expose the cultural implications, costs, and possibilities of the new knowledge and technologies.

Information Arts: Intersections of Art, Science, and Technology.
Stephen Wilson

