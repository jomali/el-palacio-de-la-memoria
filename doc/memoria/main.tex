%!TEX TS-program = pdflatex
%!TEX encoding = utf8

\documentclass[12pt, twoside]{book}
\usepackage[T1]{fontenc}
\usepackage[utf8]{inputenc}
\usepackage[english, spanish, es-noshorthands]{babel}

%% FONTS: libertine+biolinum+stix
\usepackage[mono=false]{libertine}
\usepackage[notext]{stix}

% =============================================================================
% DATOS IMPORTANTES
%

\title{TRABAJO DE FIN DE GRADO}
\author{J. Francisco Martín Lisaso}
\date{\today}
\newcommand{\tutores}[1]{\newcommand{\guardatutores}{#1}}
\tutores{Prof. Andrés Iglesias\\
         Prof. Akemi Gálvez}

% =============================================================================
% PÁGINAS DE TÍTULOS
%

\usepackage[final]{pdfpages}

\makeatletter
\edef\maintitle{\@title}
\renewcommand\maketitle{%
  % \begin{titlepage}
  %     \vspace*{1.5cm}
  %     \parskip=0pt
  %     \Huge\bfseries
  %     \begin{center}
  %         \leavevmode\includegraphics[totalheight=6cm]{sello.pdf}\\[2cm]
  %         \@title
  %     \end{center}
  %     \vspace{1cm}
  %     \begin{center}
  %         \@author
  %     \end{center}
  % \end{titlepage}
  %
  % \begin{titlepage}
  % \parindent=0pt
  % \begin{flushleft}
  % \vspace*{1.5mm}
  % \setlength\baselineskip{0pt}
  % \setlength\parskip{0mm}
  % \begin{center}
  %     \leavevmode\includegraphics[totalheight=4.5cm]{sello.pdf}
  % \end{center}
  % \end{flushleft}
  % \vspace{1cm}
  % \bgroup
  % \Large \bfseries
  % \begin{center}
  % \@title
  % \end{center}
  % \egroup
  % \vspace*{.5cm}
  % \begin{center}
  % \@author
  % \end{center}
  % \vspace*{1.8cm}
  % \begin{flushright}
  % \begin{minipage}{8.45cm}
  %     Memoria presentada como parte de los requisitos para la obtención del título de
  %     Grado en Ingeniería Informática por la Universidad de Cantabria.
  %
  %     \vspace*{7.5mm}
  %
  %     Tutorizada por
  %     % \vspace*{5mm}
  % \end{minipage}\par
  % \begin{tabularx}{8.45cm}[b]{@{}l}
  %     \guardatutores
  % \end{tabularx}
  %  \end{flushright}
  %     \vspace*{\fill}
  %  \end{titlepage}
   %%% Esto es necesario...
   \pagestyle{tfg}
   \renewcommand{\chaptermark}[1]{\markright{\thechapter.\space ##1}}
   \renewcommand{\sectionmark}[1]{}
   \renewcommand{\subsectionmark}[1]{}
  }
\makeatother

% =============================================================================
% GRÁFICOS Y DEFINICIÓN DE COLORES
%

\usepackage{tikz}
% \usepackage[pdftex]{graphicx}

% declare the path(s) where your graphic files are
\graphicspath{{./res/}}
% and their extensions so you won't have to specify these with
% every instance of \includegraphics
\DeclareGraphicsExtensions{.jpeg,.jpg,.pdf,.png,.eps}

\definecolor{UCblue}{cmyk}{1,0.09,0,0.56}
% se definen además otros tres colores: el color base 'UCblue' con brillo
% aumentado, y los dos colores ternarios adyacentes a su complementario
\definecolor{UCshinyblue}{cmyk}{0.99,0.08,0,0}
\definecolor{UCshinyred}{cmyk}{0,0.95,1,0}
\definecolor{UCshinyyellow}{cmyk}{0,0.32,1,0}

\DeclareTextFontCommand{\textbf}{\bfseries\color{UCblue}}
%\newcommand\textcolored[1]{\textcolor{UCblue}{\textbf{#1}}}

% =============================================================================
% OTROS
%

\usepackage[]{tabularx}
\usepackage[]{enumitem}
\usepackage[]{hyperref}

\setlist{noitemsep}

\hypersetup{
	bookmarks = true,
    colorlinks = false,
	citebordercolor = {UCshinyyellow},
	citecolor = {UCshinyyellow},
	filecolor = {UCblue},
    linkbordercolor = {UCblue},
	linkcolor = {UCblue},
	pdfstartview = {FitH},
	unicode = true,
	urlbordercolor = {UCshinyblue},
	urlcolor = {UCshinyblue}
}

% =============================================================================
% MATEMÁTICAS Y TEOREMAS
%

\usepackage[]{amsmath}
\usepackage[]{amsthm}
\usepackage[]{mathtools}
\usepackage[]{bm}
\usepackage[]{thmtools}
\newcommand{\marcador}{\vrule height 10pt depth 2pt width 2pt \hskip .5em\relax}
\newcommand{\cabeceraespecial}{%
    \color{UCblue}%
    \normalfont\bfseries}
\declaretheoremstyle[
    spaceabove=\medskipamount,
    spacebelow=\medskipamount,
    headfont=\cabeceraespecial\marcador,
    notefont=\cabeceraespecial,
    notebraces={(}{)},
    bodyfont=\normalfont\itshape,
    postheadspace=1em,
    numberwithin=chapter,
    headindent=0pt,
    headpunct={.}
    ]{importante}
\declaretheoremstyle[
    spaceabove=\medskipamount,
    spacebelow=\medskipamount,
    headfont=\normalfont\itshape\color{UCblue},
    notefont=\normalfont,
    notebraces={(}{)},
    bodyfont=\normalfont,
    postheadspace=1em,
    numberwithin=chapter,
    headindent=0pt,
    headpunct={.}
    ]{normal}
\declaretheoremstyle[
    spaceabove=\medskipamount,
    spacebelow=\medskipamount,
    headfont=\normalfont\itshape\color{UCblue},
    notefont=\normalfont,
    notebraces={(}{)},
    bodyfont=\normalfont,
    postheadspace=1em,
    headindent=0pt,
    headpunct={.},
    numbered=no,
    qed=\color{UCblue}\marcador
    ]{demostracion}

% Los nombres de los enunciados. Añade los que necesites.
\declaretheorem[name=Observaci\'on, style=normal]{remark}
\declaretheorem[name=Corolario, style=normal]{corollary}
\declaretheorem[name=Proposici\'on, style=normal]{proposition}
\declaretheorem[name=Lema, style=normal]{lemma}

\declaretheorem[name=Teorema, style=importante]{theorem}
\declaretheorem[name=Definici\'on, style=importante]{definition}

\let\proof=\undefined
\declaretheorem[name=Demostraci\'on, style=demostracion]{proof}

% =============================================================================
% COMPOSICIÓN DE LA PÁGINA
%

\usepackage[
	a4paper,
    textwidth=80ex,
]{geometry}

\linespread{1.069}
\parskip=10pt plus 1pt minus .5pt
\frenchspacing
% \raggedright

% =============================================================================
% COMPOSICIÓN DE LOS TÍTULOS
%

\usepackage[explicit]{titlesec}

% XXX - Para corregir un bug introducido en 'titlesec' v2.10.1
\usepackage{etoolbox}
\makeatletter
\patchcmd{\ttlh@hang}{\parindent\z@}{\parindent\z@\leavevmode}{}{}
\patchcmd{\ttlh@hang}{\noindent}{}{}{}
\makeatother
% XXX - Para corregir un bug introducido en 'titlesec' v2.10.1

\newcommand{\hsp}{\hspace{20pt}}
\titleformat{\chapter}[hang]
	{\Huge\sffamily\bfseries}
	{\thechapter\hsp\textcolor{UCblue}{\vrule width 2pt}\hsp}{0pt}
	{#1}
\titleformat{\section}
	{\normalfont\Large\sffamily\bfseries}{\thesection\space\space}
	{1ex}
	{#1}
\titleformat{\subsection}
	{\normalfont\large\sffamily}{\thesubsection\space\space}
	{1ex}
	{#1}

% =============================================================================
% CABECERAS DE PÁGINA
%
\usepackage[]{fancyhdr}
\usepackage[]{emptypage}
\fancypagestyle{plain}{%
    \fancyhf{}%
    \renewcommand{\headrulewidth}{0pt}
    \renewcommand{\footrulewidth}{0pt}
}
\fancypagestyle{tfg}{%
    \fancyhf{}%
    \renewcommand{\headrulewidth}{0pt}
    \renewcommand{\footrulewidth}{0pt}
    \fancyhead[LE]{{\normalsize\color{UCblue}\bfseries\thepage}\quad
                    \small\textsc{\MakeLowercase{\maintitle}}}
    \fancyhead[RO]{\small\textsc{\MakeLowercase{\rightmark}}%
                    \quad{\normalsize\bfseries\color{UCblue}\thepage}}%
                    }

% =============================================================================
% INICIO DEL DOCUMENTO
%

\begin{document}

% Portada:

\includepdf[pages=-]{resources/portada.pdf}
\
\thispagestyle{empty}
\cleardoublepage
\maketitle

% Agradecimientos:

\pagenumbering{gobble}	%% sin numero de pagina
% TODO

\chapter*{Agradecimientos}

Nota de agradecimiento para aquellos que han contribuido emocionalmente al proyecto de fin de Grado.

\
\thispagestyle{empty}
\cleardoublepage

%% Resumen (español):

\pagenumbering{roman}	%% numeros de pagina romanos
% TODO

\chapter*{Resumen}

Un resumen de un párrafo sobre el problema planteado en el proyecto y la solución propuesta. Alrededor de 300 palabras.
\\

\textbf{Keywords}: Ficción Interactiva; Narratología

\
\thispagestyle{empty}
\cleardoublepage

%% Resumen (inglés):

% TODO

\chapter*{Abstract}

% \addcontentsline{toc}{chapter}{English Abstract}
\markright{English Abstract}

\begin{otherlanguage}{english}
    According to the guidelines, every dissertation should include a short english abstract at the beginning. In the abstract, you describe in general terms what is your dissertation about, the main points you want to make, and any important consequences that may arise.

	\textbf{Keywords}: Interactive Fition; Narratology
\end{otherlanguage}

\
\thispagestyle{empty}
\cleardoublepage

%% Índices:

\frontmatter
\tableofcontents
\listoffigures
\listoftables
%% \listofalgorithms
\newpage

% Capítulos:

\pagenumbering{arabic}
\setcounter{page}{1}
\mainmatter
% TODO

\chapter{Introducción}\label{ch:introduccion}

\section{Una Ficción Interactiva}

Lorem ipsum dolor sit amet\cite{Montfort:2004}, consectetur adipiscing elit. Integer egestas quam ullamcorper justo lobortis convallis. Fusce sit amet erat eu sapien cursus pretium. Suspendisse vitae euismod enim. Suspendisse commodo, diam quis elementum interdum, odio diam rutrum ex, a viverra libero metus ac dolor. Quisque vitae arcu vel nibh finibus pretium. Morbi vitae interdum est. Morbi dapibus ipsum non vestibulum fringilla. Phasellus lobortis turpis lacus, quis scelerisque elit dictum nec. Suspendisse vel volutpat mauris. Curabitur at nisl magna. Vivamus porttitor lacinia mi, sit amet faucibus lacus imperdiet id. Nullam ac volutpat quam. Nullam imperdiet tincidunt metus nec convallis. Nam id tristique ex, vitae fermentum purus.

\section{Accesibilidad e inclusión digital}

Lorem ipsum\cite{Wilson:2002} dolor sit amet, consectetur adipiscing elit.

\section{Objetivos del proyecto}

Lorem ipsum dolor sit amet, consectetur adipiscing elit.

\section{Procedimiento}

Lorem ipsum dolor sit amet, consectetur adipiscing elit.

% TODO

\chapter{Herramientas}

\section{Sistema de creación Inform 6}

Lorem ipsum dolor sit amet\cite{Stanley1970}, consectetur adipiscing elit. Integer egestas quam ullamcorper justo lobortis convallis. Fusce sit amet erat eu sapien cursus pretium. Suspendisse vitae euismod enim. Suspendisse commodo, diam quis elementum interdum, odio diam rutrum ex, a viverra libero metus ac dolor. Quisque vitae arcu vel nibh finibus pretium. Morbi vitae interdum est. Morbi dapibus ipsum non vestibulum fringilla. Phasellus lobortis turpis lacus, quis scelerisque elit dictum nec. Suspendisse vel volutpat mauris. Curabitur at nisl magna. Vivamus porttitor lacinia mi, sit amet faucibus lacus imperdiet id. Nullam ac volutpat quam. Nullam imperdiet tincidunt metus nec convallis. Nam id tristique ex, vitae fermentum purus.

Duis lectus leo, elementum vel nunc a, ultrices consequat tortor. Lorem ipsum dolor sit amet, consectetur adipiscing elit. Curabitur at ex consequat, posuere lorem at, scelerisque dui. Sed at turpis egestas, elementum eros sit amet, pretium ex. Quisque imperdiet est a magna bibendum finibus. Sed vehicula imperdiet risus in laoreet. Nunc interdum porta ipsum, quis vulputate nulla rutrum sit amet. Curabitur euismod condimentum erat ut auctor. Donec euismod fermentum imperdiet. Vivamus fermentum mi finibus ipsum consectetur, ac vehicula turpis venenatis. Aenean vel commodo libero. Morbi vehicula tincidunt lorem, et fermentum nisl pretium vehicula. Nunc facilisis ut orci quis lobortis.

Pellentesque mollis nunc eu dolor blandit, quis bibendum tortor condimentum. Etiam id turpis ut est pulvinar consectetur. Donec nec felis turpis. Maecenas sed quam finibus, ultricies mi nec, rhoncus risus. Donec luctus sagittis quam a gravida. Donec cursus arcu turpis, sed imperdiet leo porttitor tempus. Donec elementum neque ut ante volutpat, sit amet tempor libero gravida.

Suspendisse quis vehicula leo, eget feugiat quam. Nam facilisis laoreet dignissim. Integer elementum eros id nibh condimentum, a molestie elit dapibus. Donec porttitor, quam id condimentum aliquet, dolor eros faucibus est, at semper tortor sapien vel orci. Etiam lobortis leo eros, sit amet efficitur leo lacinia ut. Sed purus orci, lacinia in facilisis id, tincidunt sed neque. Aenean aliquam pretium ante id consectetur. Morbi sollicitudin elementum quam nec fringilla. Ut pretium blandit libero ac volutpat. Proin eleifend nisl ut nisi commodo efficitur. Morbi dolor lacus, vehicula ut justo in, laoreet bibendum elit. Donec vestibulum in nulla sed suscipit. Donec non nulla ut dolor commodo venenatis ut vitae ligula. Proin pretium, nulla at pulvinar condimentum, dolor dolor lobortis magna, in tincidunt orci risus nec nisi. Sed placerat sagittis ante, id tempor justo vulputate blandit. Nullam suscipit finibus sem, quis viverra elit mollis a.

Duis iaculis leo non eros posuere, vel efficitur orci suscipit. Curabitur elementum tellus turpis. Quisque sed metus commodo, eleifend ante vel, vulputate mauris. Maecenas nec orci eget enim blandit sollicitudin. Phasellus nec justo ante. Mauris porttitor nunc dapibus, tincidunt tellus varius, posuere orci. Morbi diam lacus, fermentum in aliquam sed, scelerisque et odio. Sed lacinia nisl eget hendrerit gravida. Quisque scelerisque mauris nec augue interdum scelerisque. Sed hendrerit rutrum quam, sit amet pulvinar quam finibus at. Duis id mauris et dui pulvinar rhoncus ac aliquet ex. Phasellus eros ligula, molestie sit amet vehicula in, congue facilisis metus. Aliquam sem est, dapibus finibus tempus eget, dapibus eget tortor. Vestibulum mollis malesuada eros ac finibus.

Curabitur in fermentum ipsum, at ullamcorper arcu. Donec ut porttitor sapien, pulvinar dapibus lacus. Nunc sodales tincidunt velit, eget sollicitudin est semper eget. Fusce aliquet nunc imperdiet arcu pharetra, in blandit urna consequat. Quisque vel cursus ante. Ut aliquam, magna ut bibendum rutrum, justo nibh dignissim elit, at bibendum magna orci egestas libero. Vivamus vitae mollis nisl, ut rutrum ligula. Aenean ac urna pellentesque ex varius venenatis. Duis vel lorem augue. Pellentesque a imperdiet dui, et blandit diam. Aliquam quis turpis eu velit volutpat facilisis sed quis mi.

Vestibulum a porttitor velit. Ut quam tellus, semper id tempor sit amet, dictum non arcu. Ut semper elementum mauris eget condimentum. Proin tristique, enim ut porttitor tristique, justo risus sollicitudin sem, quis placerat sem velit sed metus. Aliquam ornare diam ac risus porta, quis varius odio efficitur. Phasellus luctus posuere eleifend. Maecenas vel sem vel risus tristique suscipit vel vel eros. Aliquam erat dolor, tristique quis dui eu, volutpat hendrerit sem. Proin augue ex, sodales ac semper a, dignissim a mi. Curabitur eu lacus maximus, porta neque a, feugiat elit. Fusce eu posuere enim. Nam et leo vitae turpis vehicula vulputate. Donec odio dolor, lobortis sed elit at, volutpat imperdiet mi. Etiam eu nisl eleifend, viverra leo sed, tincidunt dolor. Curabitur feugiat vel ante at luctus. Maecenas orci arcu, luctus et maximus sed, facilisis ac quam.

Vestibulum fermentum mauris augue, sit amet elementum velit lobortis at. Fusce non finibus ante. Quisque eu leo vel metus ullamcorper dictum. Maecenas molestie enim at porta sodales. Integer ac porta dolor. Curabitur et porta risus, vitae blandit metus. Mauris gravida blandit ex, vel iaculis eros elementum vitae. Nullam ac lacus eleifend, porta orci nec, fermentum velit. Cras a molestie dui. Nulla ornare turpis orci, sed accumsan odio tempus at.

\begin{theorem}[Euclides]\label{thm:th1}
    Esto es un Teorema. Se numeran a partir del 1 en cada capítulo. Como son importantes, tienen un cuadrado rojo al principio. Llevan letra cursiva.
\end{theorem}

\begin{proof}
    Esto es la demostración. Al final de la demostración se puede ver un cuadrado rojo similar al de los teoremas. Las demostraciones no llevan letra cursiva.
\end{proof}

\begin{definition}\label{def:1}
    Esto es una definición. Las definiciones son importantes; también llevan un cuadradito rojo.
\end{definition}

\subsection{Las máquinas virtuales Z-Machine y Glulx}

\begin{remark}
    Esto es una observación, que dice que $e=mc^{2}$. Como las observaciones no son importantes, no llevan cuadrado rojo, y el tipo de letra no es cursiva.
\end{remark}

\begin{proof}
    Si la demostración acaba en una fórmula, para poner el cuadrado rojo a la altura de la última formula, hay que usar la orden \verb|\qedhere|, como en este caso:
    \[
        e=mc^{2}.\qedhere
    \]
\end{proof}

\begin{corollary}\label{cor:1}
    Esto es un corolario.
\end{corollary}

\begin{proposition}\label{pro:1}
    Esto es una proposición.
\end{proposition}

\begin{lemma}[Gauss]\label{lem:1}
    Esto es un lema.
\end{lemma}

\section{Extensión gráfica GWindows}

Lorem ipsum dolor sit amet, consectetur adipiscing elit. Integer egestas quam ullamcorper justo lobortis convallis. Fusce sit amet erat eu sapien cursus pretium. Suspendisse vitae euismod enim. Suspendisse commodo, diam quis elementum interdum, odio diam rutrum ex, a viverra libero metus ac dolor. Quisque vitae arcu vel nibh finibus pretium. Morbi vitae interdum est. Morbi dapibus ipsum non vestibulum fringilla. Phasellus lobortis turpis lacus, quis scelerisque elit dictum nec. Suspendisse vel volutpat mauris. Curabitur at nisl magna. Vivamus porttitor lacinia mi, sit amet faucibus lacus imperdiet id. Nullam ac volutpat quam. Nullam imperdiet tincidunt metus nec convallis. Nam id tristique ex, vitae fermentum purus.

\section{Limitaciones de la librería por defecto}

Lorem ipsum dolor sit amet, consectetur adipiscing elit. Integer egestas quam ullamcorper justo lobortis convallis. Fusce sit amet erat eu sapien cursus pretium. Suspendisse vitae euismod enim. Suspendisse commodo, diam quis elementum interdum, odio diam rutrum ex, a viverra libero metus ac dolor. Quisque vitae arcu vel nibh finibus pretium. Morbi vitae interdum est. Morbi dapibus ipsum non vestibulum fringilla. Phasellus lobortis turpis lacus, quis scelerisque elit dictum nec. Suspendisse vel volutpat mauris. Curabitur at nisl magna. Vivamus porttitor lacinia mi, sit amet faucibus lacus imperdiet id. Nullam ac volutpat quam. Nullam imperdiet tincidunt metus nec convallis. Nam id tristique ex, vitae fermentum purus.

% TODO

\chapter{Extensiones a la librería Inform}\label{ch:extensiones}

\section{Interfaz biplataforma para la selección de estilos de texto}\label{sec:textStyles}

La selección del estilo con el que se imprime texto en el sistema Inform no es un aspecto específico de su lenguaje de programación sino que está estrechamente ligado a la máquina virtual sobre la que se ejecuta el software. Así, tanto las rutinas de selección de estilo como el propio número de estilos de texto disponibles son diferentes en Máquina-Z y Glulx. El objetivo de esta extensión a la librería es ofrecer al autor de ficción interactiva una interfaz que le permita abstraerse de la plataforma concreta para la que está desarrollando la aplicación.

En Máquina-Z existe un total de cinco estilos de texto diferentes (algunos intérpretes pueden reconocer combinaciones entre ellos, pero éstas no forman parte del estándar)\cite{Nelson:Fillmore:2014}. En Glulx, por su parte, el conjunto de estilos disponibles se extiende hasta los once\cite{Plotkin:2017:a}. Como el número de estilos disponibles entre ambas máquinas virtuales es diferente, la librería Inform (biplataforma) establece una función sobreyectiva de correspondencia entre ellas\footnote{Según esta función de correspondencia, cuando la librería utiliza un estilo en Glulx que no está definido en Máquina-Z, como por ejemplo \emph{Titular}, \emph{Subtitular} o \emph{Alerta}, en ésta última se emplea simplemente \emph{Negrita}.}. A la hora de determinar los estilos de texto básicos de la extensión, utilizaremos como base la especificación de estilos de Glulx y las correspondencias con los estilos de Máquina-Z que se establecen en la librería Inform (ver cuadro\ref{table:text-styles-estilos-basicos}).

\begin{table}[]
\centering
\begin{tabular}{llll}
\hline
\textbf{\#} &\textbf{Text Styles} & \textbf{Estilo Glulx} & \textbf{Estilo Máquina-Z} \\ \hline
$0$		& Upright		& Normal		& Normal		\\
$1$		& Stressed		& Emphasized	& Italic		\\
$2$		& Important		& Subheader		& Bold			\\
$3$		& Monospaced	& Preformatted	& Fixed-width	\\
$4$		& Header		& Header		& Bold			\\
$5$		& Note			& Note			& Bold e Italic	\\
$6$		& Reversed		& Alert			& Reverse		\\
$7$		& Quote			& BlockQuote	& Fixed-width	\\
$8$		& Input			& Input			& Bold			\\
$9$		& User1			& User1			& Normal		\\
$10$	& User2			& User2			& Normal		\\ \hline
\end{tabular}
\caption{Estilos de \emph{textStyles} y sus correspondencias en Glulx y Máquina-Z}
\label{table:text-styles-estilos-basicos}
\end{table}

\subsection{Reglas de impresión contextuales y estilo especial para mensajes extradiegéticos}

Como complemento a los estilos básicos, la extensión define dos reglas de impresión contextuales\footnote{Una regla de impresión, en Inform, es simplemente una rutina de un único parámetro que se encarga de imprimir los datos que se le pasan de una determinada manera\cite{Firth:2006}.} para enfatizar y destacar, que seleccionarán el estilo de texto a utilizar en función del estilo utilizado en ese momento.

Se implementa también un estilo adicional para ser utilizado por los mensajes de tipo extradiegético emitidos por el sistema para facilitar la interacción con la obra, pero que no tienen que ver con el resto de mensajes del narrador. Este estilo contará también con su propia regla de impresión.


\section{Marco para facilitar la interacción por hipervínculos}

La máquina virtual Glulx, al incluir soporte nativo para la librería de entrada/salida Glk, ofrece un conjunto de funcionalidades adicionales a aquellas de Máquina-Z entre las que se encuentra la posibilidad de responder a eventos de selección de hipervínculos\ref{Plotkin:2017:b}.

El propósito de \textbf{hyperlinks} es ofrecer la infreaestructura para que el autor de ficción interactiva pueda incluir de un modo sencillo interacción por medio de hipervínculos en sus obras compiladas para Glulx. Aunque la funcionalidad de la extensión es exclusiva de Glulx, a fin de facilitar el desarrollo de obras biplataforma, \textbf{hyperlinks} puede utilizarse tanto en Glulx como en Máquina-Z ---al compilar para esta segunda máquina virtual simplemente no se dispondrá de la funcionalidad relacionada con los hipervínculos---.


\section{Aproximación de interfaz gráfica adaptativa con `GWindows'}

Si bien el modelo de interfaz gráfica utilizado tanto por Máquina-Z como por Glulx es ciertamente restrictivo, la segunda permite una mayor libertad para crear y manipular las diferente ventanas que conforman la interfaz.

Utilizando las funcionalidades de la extensión `GWindows' de Ross L. se define una interfaz con una serie de ventanas que pueden cambiar de tamaño en función de las dimensiones de la ventana padre. Así podemos adaptar el tamaño de la ventana en que se imprime el texto.


\section{Sistema para hacer selecciones entre un conjunto de opciones}

La extensión \textbf{ChoiceSet} permite implementar sistemas de conversación con inventario de temas ---y, en general, cualquier otro sistema que utilice un mecanismo de selección de opciones de entre un conjunto dado--- utilizando análisis no estricto de la entrada de usuario para el reconocimiento de patrones.

Está basada en las notas de Eric Eve sobre sistemas de conversación\cite{Eve:2008} y construída sobre las extensiones NPC\_Conversacion v1.0 de Mastodon, y topicInventory v2.1 ---choiceSet es una generalización actualizada sobre ésta última, propia---.

\textbf{ChoiceSet} permite listar ante el usuario un conjunto de opciones de entre las que puede seleccionar una introduciendo alguna de las palabras clave de la opción; una funcionalidad que facilita, por ejemplo, la construcción de CONVERSACIONES utilizando un sistema de INVENTARIO DE TEMAS. Estos sistemas de conversación están ideados con el objetivo de esquivar los problemas que suelen presentar los sistemas de conversación más habituales en los relatos interactivos; dificultades para adivinar la palabra clave en sistemas basados en acciones ASK/TELL, simplificación excesiva en sistemas basados en la acción TALK TO, o falta de libertad y ruptura de la interfaz textual en sistemas de menús.

Al utilizar un inventario de temas, en esencia, se presenta al usuario una lista de temas de conversación sugeridos basándose en el conocimiento del personaje protagonista. De esta forma, el usuario puede hacer referencia a los temas que quiera lanzar utilizando una interfaz similar a la de los sistemas ASK/TELL, pero sin necesidad de tener que adivinar por sí mismo la palabra clave que lanza el tema.

\subsection{Análisis no estricto de la entrada de usuario}

El reconocimiento de patrones entre la entrada de usuario y las palabras clave de cada elección se lleva a cabo por medio de análisis no estricto; si entre todas las palabras introducidas por el usuario se reconocen claves de una o más elecciones, se lanza aquella con más coincidencias, ignorando el resto de la entrada.


\section{Mensajes de la librería en español con flexiones gramaticales de persona y tiempo, género y número}

La librería Inform-INFSP, con el conjunto de mensajes por defecto para la entrada de usuario, sólo contempla mensajes en segunda persona del presente, masculino y singular.

Se ha reescrito por completo para incluir la posibilidad de modificar en tiempo de ejecución tanto el tiempo verbal de los mensajes (presente, pasado, futuro) y la persona (primera, segunda, tercera), como el género (masculino, femenino) y el número (singular, plural) ---estos últimos, en función de la definición del objeto 'player' controlado por el usuario.

Además, si bien los mensajes están basados en gran medida en los mensajes por defecto de la librería original, se han incluido modificaciones para diferenciar los mensajes diegéticos (del narrador, relacionados con el relato) por aquellos extradiegéticos (del sistema, relacionados con la interacción con el software). Además, se han modificado varios mensajes que presentaban ciertas inconsistencias de estilo o tono ---heredados de las aventuras clásicas de Infocom--- por otros más neutros, siguiendo en parte el modelo de la extensión de Inform 7, \texttt{Neutral Library Messages}\cite{Reed:2007}.


\section{Especificación de requisitos}

Lorem ipsum dolor sit amet, consectetur adipiscing elit.

\section{Diseño de alto nivel}

Lorem ipsum dolor sit amet, consectetur adipiscing elit.

\section{Diseño detallado}

Lorem ipsum dolor sit amet, consectetur adipiscing elit.

% TODO

\chapter{El Palacio de la Memoria}\label{ch:palacio-memoria}

\section{Planteamiento}

Lorem ipsum dolor sit amet, consectetur adipiscing elit. Integer egestas quam ullamcorper justo lobortis convallis. Fusce sit amet erat eu sapien cursus pretium. Suspendisse vitae euismod enim. Suspendisse commodo, diam quis elementum interdum, odio diam rutrum ex, a viverra libero metus ac dolor. Quisque vitae arcu vel nibh finibus pretium. Morbi vitae interdum est. Morbi dapibus ipsum non vestibulum fringilla. Phasellus lobortis turpis lacus, quis scelerisque elit dictum nec. Suspendisse vel volutpat mauris. Curabitur at nisl magna. Vivamus porttitor lacinia mi, sit amet faucibus lacus imperdiet id. Nullam ac volutpat quam. Nullam imperdiet tincidunt metus nec convallis. Nam id tristique ex, vitae fermentum purus.

\section{Exploración secuencial y exploración abierta}

Lorem ipsum dolor sit amet, consectetur adipiscing elit. Integer egestas quam ullamcorper justo lobortis convallis. Fusce sit amet erat eu sapien cursus pretium. Suspendisse vitae euismod enim. Suspendisse commodo, diam quis elementum interdum, odio diam rutrum ex, a viverra libero metus ac dolor. Quisque vitae arcu vel nibh finibus pretium. Morbi vitae interdum est. Morbi dapibus ipsum non vestibulum fringilla. Phasellus lobortis turpis lacus, quis scelerisque elit dictum nec. Suspendisse vel volutpat mauris. Curabitur at nisl magna. Vivamus porttitor lacinia mi, sit amet faucibus lacus imperdiet id. Nullam ac volutpat quam. Nullam imperdiet tincidunt metus nec convallis. Nam id tristique ex, vitae fermentum purus.

\subsection{Recuerdos interactivos en forma de flashback}

Lorem ipsum dolor sit amet, consectetur adipiscing elit.

\section{Aleatoriedad}

Lorem ipsum dolor sit amet, consectetur adipiscing elit. Integer egestas quam ullamcorper justo lobortis convallis. Fusce sit amet erat eu sapien cursus pretium. Suspendisse vitae euismod enim. Suspendisse commodo, diam quis elementum interdum, odio diam rutrum ex, a viverra libero metus ac dolor. Quisque vitae arcu vel nibh finibus pretium. Morbi vitae interdum est. Morbi dapibus ipsum non vestibulum fringilla. Phasellus lobortis turpis lacus, quis scelerisque elit dictum nec. Suspendisse vel volutpat mauris. Curabitur at nisl magna. Vivamus porttitor lacinia mi, sit amet faucibus lacus imperdiet id. Nullam ac volutpat quam. Nullam imperdiet tincidunt metus nec convallis. Nam id tristique ex, vitae fermentum purus.

\section{Agencia}

Lorem ipsum dolor sit amet, consectetur adipiscing elit. Integer egestas quam ullamcorper justo lobortis convallis. Fusce sit amet erat eu sapien cursus pretium. Suspendisse vitae euismod enim. Suspendisse commodo, diam quis elementum interdum, odio diam rutrum ex, a viverra libero metus ac dolor. Quisque vitae arcu vel nibh finibus pretium. Morbi vitae interdum est. Morbi dapibus ipsum non vestibulum fringilla. Phasellus lobortis turpis lacus, quis scelerisque elit dictum nec. Suspendisse vel volutpat mauris. Curabitur at nisl magna. Vivamus porttitor lacinia mi, sit amet faucibus lacus imperdiet id. Nullam ac volutpat quam. Nullam imperdiet tincidunt metus nec convallis. Nam id tristique ex, vitae fermentum purus.

% TODO

\chapter{Conclusiones y trabajo futuro}

\section{Conclusiones}

Lorem ipsum dolor sit amet, consectetur adipiscing elit. Integer egestas quam ullamcorper justo lobortis convallis. Fusce sit amet erat eu sapien cursus pretium. Suspendisse vitae euismod enim. Suspendisse commodo, diam quis elementum interdum, odio diam rutrum ex, a viverra libero metus ac dolor. Quisque vitae arcu vel nibh finibus pretium. Morbi vitae interdum est. Morbi dapibus ipsum non vestibulum fringilla. Phasellus lobortis turpis lacus, quis scelerisque elit dictum nec. Suspendisse vel volutpat mauris. Curabitur at nisl magna. Vivamus porttitor lacinia mi, sit amet faucibus lacus imperdiet id. Nullam ac volutpat quam. Nullam imperdiet tincidunt metus nec convallis. Nam id tristique ex, vitae fermentum purus.

\section{Trabajo futuro}

Lorem ipsum dolor sit amet, consectetur adipiscing elit. Integer egestas quam ullamcorper justo lobortis convallis. Fusce sit amet erat eu sapien cursus pretium. Suspendisse vitae euismod enim. Suspendisse commodo, diam quis elementum interdum, odio diam rutrum ex, a viverra libero metus ac dolor. Quisque vitae arcu vel nibh finibus pretium. Morbi vitae interdum est. Morbi dapibus ipsum non vestibulum fringilla. Phasellus lobortis turpis lacus, quis scelerisque elit dictum nec. Suspendisse vel volutpat mauris. Curabitur at nisl magna. Vivamus porttitor lacinia mi, sit amet faucibus lacus imperdiet id. Nullam ac volutpat quam. Nullam imperdiet tincidunt metus nec convallis. Nam id tristique ex, vitae fermentum purus.


% Bibliografía:

\backmatter
\bibliographystyle{apalike}
\bibliography{referencias}

\end{document}
